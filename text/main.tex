\documentclass[12pt]{article} % Document class

% Packages for math symbols and formatting
\usepackage{amsmath, amssymb, amsthm} 
\usepackage{graphicx} % For including graphics
\usepackage{hyperref} % For hyperlinks

% Theorem, Lemma, and Proof Environments
\newtheorem{theorem}{Theorem}
\newtheorem{lemma}{Lemma}
\newtheorem{corollary}{Corollary}
\newtheorem{definition}{Definition}

\theoremstyle{remark}
\newtheorem*{remark}{Remark}

% Title and Author
\title{Notes on Nonparametric Statistics}
\author{José Manuel Rodríguez Caballero}
\date{\today}

\begin{document}

\maketitle % Generates title, author, and date

\section{The Wilcoxon Rank Sum}
Given two positive integers $n$ and $N$, we define the Wilcoxon Rank Sum distribution $\mathbf{P}\left( W^N_n = k\right)$ via the generating function
\begin{equation}
\frac{\binom{N}{n}_q}{\binom{N}{n}} = \sum_{k=\frac{n(n+1)}{2}}^{\frac{n(2N-n+1)}{2}}  \mathbf{P}\left( W^N_n = k\right) q^k, \label{eq:binom_P}
\end{equation}
where the polynomials $\binom{N}{n}_q$ are given by the product
\begin{equation}
\prod_{n=1}^N \left( 1 - z q^n\right) = \sum_{n=0}^N (-1)^n  \binom{N}{n}_q z^n. \label{eq:def_qbinom}
\end{equation}

\begin{theorem}
For two positive integers $n$ and $N$, we have
\begin{eqnarray}
\label{eq:expectation_W} \mathbf{E}\left( W^N_n \right) &=& \frac{n(N+1)}{2}, \\
\label{eq:variance_W} \mathbf{Var}\left( W^N_n \right) &=& \frac{n(N-n)(N+1)}{12}.
\end{eqnarray}
\end{theorem}

\begin{proof}
Using the notations
\begin{eqnarray}
F &=& \prod_{n=1}^N \left( 1 - z q^n\right), \\
G &=& \sum_{k\geq 1} \left(q^k + q^{2k} + q^{3k} + ... + q^{Nk}  \right) \frac{z^k}{k},
\end{eqnarray}
we have
\begin{equation}
F = e^{-G}.
\end{equation}

Let $\mathcal{H} = q\frac{d}{dq}$, it follows
\begin{eqnarray}
\mathcal{H} F &=& - e^{-G} \mathcal{H}, \\
\mathcal{H}^2 F &=& e^{-G} \left[ \left(\mathcal{H} G\right)^2 - \mathcal{H}^2 G \right] .
\end{eqnarray}

We compute 
\begin{eqnarray}
\mathcal{H} G \big|_{q=1} &=& \frac{N(N+1)}{2} \frac{z}{1-z}, \\
\mathcal{H}^2 G \big|_{q=1} &=& \frac{N(N+1)(2N+1)}{6} \frac{z}{(1-z)^2},
\end{eqnarray}
and
\begin{eqnarray}
\mathcal{H} F \big|_{q=1} &=& -\frac{N(N+1)}{2} z(1-z)^{N-1}, \\
\mathcal{H}^2 F \big|_{q=1} &=& \scalebox{1}{$(1-z)^{N-2}\left[ \left(\frac{N(N+1)}{2} \right)^2 z^2 - \frac{N(N+1)(2N+1)}{6} z  \right].$}
\end{eqnarray}

Considering the coefficients of both sides,
\begin{eqnarray}
\label{eq:expectation_W1} \frac{\mathcal{H} \left.\binom{N}{n}_q\right|_{q=1}}{\binom{N}{n}} &=& \frac{n(N+1)}{2}, \\
\label{eq:expectation_W2} \frac{\mathcal{H}^2 \left.\binom{N}{n}_q\right|_{q=1}}{\binom{N}{n}} &=& \frac{n(N+1)(3nN+2n+N)}{12}.
\end{eqnarray}

By definition of the expected value,
\begin{eqnarray}
\label{eq:expectation_W3} \frac{\mathcal{H} \left.\binom{N}{n}_q\right|_{q=1}}{\binom{N}{n}} &=& \mathbf{E}\left[W^N_n\right], \\
\label{eq:expectation_W4} \frac{\mathcal{H}^2 \left.\binom{N}{n}_q\right|_{q=1}}{\binom{N}{n}} &=& \mathbf{E}\left[\left(W^N_n\right)^2\right].
\end{eqnarray}

Combining \eqref{eq:expectation_W1} and \eqref{eq:expectation_W3}, we obtain \eqref{eq:expectation_W}.

By definition of the variance,

\begin{equation}
\label{eq:variance_W1}
\mathbf{Var}\left[W^N_n\right] = \mathbf{E}\left[\left(W^N_n\right)^2\right] - \left(\mathbf{E}\left[W^N_n\right] \right)^2.
\end{equation}

Combining \eqref{eq:expectation_W1}, \eqref{eq:expectation_W2}, \eqref{eq:expectation_W3} and \eqref{eq:variance_W1} we obtain \eqref{eq:variance_W}.
\end{proof}

\begin{theorem}
For two positive integers $n$ and $N$, we have that $W^N_n$ is symmetric, i.e.,
\begin{equation}
\label{eq:sym_W} \mathbf{P}\left( W^N_n = k \right) = \mathbf{P}\left( W^N_n = n\left(N+1\right) - k \right),
\end{equation}
for all $\frac{n(n+1)}{2} \leq k \leq \frac{n(2N-n+1)}{2}$.
\end{theorem}

\begin{proof}
The function $F(q,z) = \prod_{n=1}^N \left( 1 - z q^n\right)$ is invariant under the transformation $q\mapsto q^{-1}$ and $z\mapsto z q^{N+1}$,

\begin{equation} \label{eq:sym_F}
F\left(q^{-1}, z q^{N+1}\right) = F\left(q, z\right).
\end{equation}

Equating coefficients in \eqref{eq:sym_F}, according to \eqref{eq:def_qbinom}, we obtain

\begin{equation} \label{eq:sym_G}
q^{n(N+1)} \binom{N}{n}_{q^{-1}} = \binom{N}{n}_q,
\end{equation}
where $\binom{N}{n}_{q^{-1}}$ is the result of the substitution of $q$ by $q^{-1}$ in $\binom{N}{n}_q$. Combining \eqref{eq:sym_G} and \eqref{eq:binom_P}, we conclude \eqref{eq:sym_W}.
\end{proof}

\end{document}
